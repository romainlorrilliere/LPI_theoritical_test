
\documentclass[A4,7pt]{article}
 

\usepackage{amsmath,amsfonts,amssymb,amsthm,latexsym}
\usepackage[english]{babel}   %% les accents dans le fichier.tex


\begin{document}

\title{What is the mathematical meaning of the species index $I_t$ of
  LPI ?} 
\date{\today}
\author{Romain Lorrilliere}

 \maketitle

 \section{The species index $I_t$  of LPI}
We calculate the species index $I_t$ following \cite{loh_living_2005,collen_monitoring_2009} methods. we implement a chain method.

First, we calculated the logarithm of the ratio of population $i$
 measure for successive years ($d_t$)
 \begin{equation}
   \label{eq:dt}
   d_{i,t} = log_{10}\left(\frac{N_{i,t}}{N_{i,t-1}}\right)
 \end{equation}
 Where $N$ is the population measure and $t$ is the year.

 Then for each species we calculate de mean of $\bar{d}_t$ across all
 the populations.

 
 \begin{equation}
   \label{eq:d_bar}
   \bar{d}_t=\frac{1}{n_t}\sum_{i=1}^{n(t)}d_{i,t}
 \end{equation}

We finally calculate the species index $I_t$ for the year $t$ as
following :
\begin{equation}
  \label{eq:I}
  I_t = I_{t-1}10^{\bar{d}_t}
\end{equation}


\section{Reformulation of $I_t$ }

We consider a simple case, all the populations of a species have the
same timely constant growth rate $\lambda_i$ 

\begin{equation}
  \label{eq:lambda}
  \lambda_i = \frac{N_{i,t}}{N_{i,t-1}}
\end{equation}

So 

\begin{equation}
  \label{eq:dt2}
   d_{i,t} = log_{10}\left(\lambda_i\right)
\end{equation}


As all population same growth rate

\begin{equation}
  \label{eq:d_bar2}
    \bar{d}_t = d_{i,t} = log_{10}\left(\lambda_i\right)
\end{equation}

For $t=1$

\begin{equation}
  \label{eq:I1}
  I_1 = 1 \times 10^{log_{10}\left(\lambda_i\right)} = \lambda_i
\end{equation}

For $t=2$

\begin{equation}
  \label{eq:I2}
  I_2 = 1 \times 10^{log_{10}\left(\lambda_i\right)} \times
    10^{log_{10}\left(\lambda_i\right)} =
    \left(10^{log_{10}\left(\lambda_i\right)}\right)^2 = \lambda_i^2
\end{equation}

For $t$


\begin{equation}
  \label{eq:It}
  I_t = \left(10^{log_{10}\left(\lambda_i\right)}\right)^t = \lambda_i^t
    \end{equation}

Finally we find that $I_t$ is a cumulative growth rate 

\begin{equation}
  \label{eq:It}
  I_t  = \lambda_i^t = \left( \frac{N_t}{N_0} \right)^t
    \end{equation}


  
    \bibliographystyle{apalike}
    \bibliography{LPI}

\end{document}

